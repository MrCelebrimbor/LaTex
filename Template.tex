\documentclass[12pt]{article}   	
\usepackage{geometry}                	
\geometry{a4paper,
			%paperheight=25cm,paperwidth=15cm,
			right=4cm,left=2cm,top=1cm,bottom=1cm,
			includeheadfoot,landscape=false
,showframe%是否显示边框
}
 %支持中文
\usepackage{xeCJK}
%各级标题格式设置
\usepackage{titlesec}
\titleformat{\chapter}[display]
{\normalfont\huge\bfseries}{\chaptertitlename\ \thechapter}{20pt}{\Huge}
\titleformat{\section}[hang]
{\large}
{%设置label的格式
\enspace 第 \thesection 章}
{0pt}%label和body的距离
{\Large}%前命令
[]%后命令
\titleformat{\subsection}
{\normalfont\large\bfseries}{\thesubsection}{1em}{}
\titleformat{\subsubsection}
{\normalfont\normalsize\bfseries}{\thesubsubsection}{1em}{}
\titleformat{\paragraph}[runin]
{\normalfont\normalsize\bfseries}{\theparagraph}{1em}{}
\titleformat{\subparagraph}[runin]
{\normalfont\normalsize\bfseries}{\thesubparagraph}{1em}{}
%各级标题四周距离设置
\titlespacing*{\chapter} {0pt}{50pt}{40pt}[0pt]
\titlespacing*{\section} {1em}{3.5ex plus 1ex minus .2ex}{2.3ex plus .2ex}[1em]
\titlespacing*{\subsection} {0pt}{3.25ex plus 1ex minus .2ex}{1.5ex plus .2ex}[0pt]
\titlespacing*{\subsubsection}{0pt}{3.25ex plus 1ex minus .2ex}{1.5ex plus .2ex}[0pt]
\titlespacing*{\paragraph} {0pt}{3.25ex plus 1ex minus .2ex}{1em}[0pt]
\titlespacing*{\subparagraph} {\parindent}{3.25ex plus 1ex minus .2ex}{1em}[0pt]

%图表标题设置
\usepackage{caption}
\captionsetup{font={small,bf},labelfont=bf} %设置图表标题的格式
\renewcommand{\tablename}{\normalsize 表}%修改英文table 为中文  表
\renewcommand{\figurename}{\normalsize 图}%修改英文figure 为中文 图

%目录
\renewcommand{\contentsname}{目录} %修改英文目录名称为中文  下同
\renewcommand{\listfigurename}{插\ 图\ 目\ 录}
\renewcommand{\listtablename}{表\ 格\ 目\ 录} 


%页眉高度设置
\setlength{\headheight}{1.5cm}%页眉高度
\setlength{\headsep}{1mm}%页眉和版心距离
%边注设置
\setlength{\marginparsep}{3mm}% 版心和边注的距离
\setlength{\marginparwidth}{2cm}% 边注的宽度
%脚注设置
\setlength{\footskip}{10mm}%页脚基线到正文基线的距离
\setlength{\footnotesep}{12pt}%脚注横线到正文最后一行基线的距离
%logo相关
\usepackage{graphicx}
\newsavebox{\mylogo}
\sbox{\mylogo}{\includegraphics[height=1.3cm]{pic/logo.png}}

%页脚页眉
\usepackage{fancyhdr}
\fancyhead[L]{\usebox{\mylogo}}%页眉左
\fancyhead[C]{}%页眉中
\fancyhead[R]{测试页面}%页眉右
\fancyfoot[L]{}%页脚左
\fancyfoot[C]{第 \thepage 页}%页脚中
\fancyfoot[R]{}%页脚右
\renewcommand{\headrulewidth}{0.4pt}%页眉横线
\renewcommand{\footrulewidth}{0.4pt}%页脚横线

\renewcommand{\baselinestretch}{1.5}%行距

%文档开始
\begin{document}
%封面
\begin{titlepage}
\vspace*{1cm}
\begin{center}
{\LARGE 标题 }\\[5cm]
{\normalsize 测试}\\[3cm]
\begin{minipage}[t]{3cm}
\centering
\end{minipage}\\[3cm]
{\large 2020年12月12日}
\end{center} 

\end{titlepage}
%显示目录
\tableofcontents % 目录
\newpage
\listoffigures% 图像目录
\newpage
\listoftables% 表格目录
\newpage
%正文
\pagestyle{fancy}
\section{测试1}
\subsection{测试11}

\newpage
\section{测试2}
\subsection{测试21}

\newpage
\section{测试3}
\subsection{测试31}
\newpage
\begin{figure}[h]
\begin{center}
\includegraphics[width=0.50\textwidth]{pic/logo.png}
\caption{测试图片1}
\label{pic1}
\end{center}
\end{figure}
\begin{figure}[h]
    \begin{center}
    \includegraphics[width=0.50\textwidth]{pic/logo.png}
    \caption{测试图片2}
    \label{pic2}
    \end{center}
    \end{figure}

\newpage
\begin{table}[h]
	\caption{测试表格1}
    \label{testtable1}
	\centering
	\begin{tabular}{cc}
		\hline
		a & a \\
		b & b
	\end{tabular}
\end{table}
\begin{table}[h]
    \caption{测试表格2}
    \label{testtable2}
	\centering
	\begin{tabular}{cc}
		\hline
		a & a \\
		b & b
	\end{tabular}
\end{table}
\end{document}  